\documentclass[a4paper,10pt]{scrartcl}
\usepackage{ngerman}
\usepackage{amsmath, amsfonts, epsfig, xspace}
\usepackage[utf8x]{inputenc}
\usepackage[english]{babel}

\usepackage{color}

\usepackage{natbib} %schönere Zitate

\usepackage[colorinlistoftodos,textsize=tiny]{todonotes}

\presetkeys{todonotes}{fancyline}{}
\definecolor{todoorange}{rgb}{1, 0.8, 0.4}

\newcommand{\comment}[1]{
\todo[bordercolor=todoorange!80!black,color=todoorange]{\textbf{Comment:} #1}
}

\newcommand{\cn}{$^\text{[citation needed] }$}

\author{Jonathan Oberländer}

\title{Automatic Detection of Linguistic Quality Violations}

\begin{document}

\maketitle

%TODO: Stichworte zu allem
% nonword "missing_spaces" nennen (?)
% labels an den Graph
% in Beispielen Probleme fett machen
% ungrammatical: 2 Sätze zusammengeklebt ODER was aus langem Satz entfernt, ebenfalls markieren
% precision & recall, etc "methods" -> "metrics"

% probieren: known words stemmen und zu testendes auch
% "NER" mit YAGO oder Wikipedia ?

% exp 2 und 2a: weglassen? auf dev-1 einzeln evaluieren

% precision/recall/…: umgekehrt ebenfalls, d.h. 
% von wie vielen Sätzen sind welche Klassen

% in Tabellen: auf was getestet?

% baseline: random bzw. most-common class

% POS-Tags-threshold, d.h. unseen n-grams -> ungrammatical

% (english resource grammar)

% mapping to source sentences
% Levenshtein auf Wortebene (?)

\section{Introduction}
% Motivation, What am I trying to do

\section{Related Work}
% other works on grammaticality and so on

\section{The LQVCorpus}
% describe the corpus and its flaws

\section{Experiments}
% All experiments, intermixed with small discussion segments
\subsection{Datelines}
% Regex foo, explanation plus evaluation(?)
\subsubsection{Method}
\subsubsection{Evaluation}

\subsection{Grammaticality}
\subsubsection{Method}
\subsubsection{Evaluation}
% previous tries with LMs(?), unknown_tokens, unknown_tokens+LM+weka->decision trees
\subsection{Redundancy}
% token overlap
%TODO: phrase based

\subsection{Unrelatedness} %evtl


%\section{Evaluation} oder auch nicht
% or skip entirely and put in experiments? final results here?

\section{Discussion}
% What seem to be the best methods for the various tasks and why?
% Why does what works work and not that which doesn't?

\section{Conclusion} % & Future Work
% What can my results be used for? What should be investigated later?
% What could possibly be needed to work on the types not looked at here? (entity, semrel, …)

\newpage
\bibliography{referenzen}
\bibliographystyle{apalike}

\end{document}